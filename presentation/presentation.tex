\documentclass[11pt]{beamer}
\usetheme{Rochester}
\usepackage{amsmath} %основной пакет для формул
\usepackage[utf8]{inputenc}
\usepackage[english,russian]{babel}
\usepackage[OT1]{fontenc}
\usepackage{amsfonts}
\usepackage{amssymb}
\usepackage{here}
\beamertemplatenavigationsymbolsempty
\author{Дьячков Вадим Вадимович\\
Жуйков Артём Александрович\\
Ламтев Антон Юрьевич\\
Леженин Юрий Игоревич\\}
\title[Алгоритмы сжатия изображений]{Алгоритмы сжатия изображений}
\setbeamercovered{transparent} 
%\setbeamertemplate{navigation symbols}{} 
%\logo{} 
\institute{Санкт-Петербургский политехнический университет Петра Великого\\
Институт компьютерных наук и технологий\\
Кафедра компьютерных систем и программных технологий\\
Группа 13501/4} 
\date{\the\year}
\begin{document}


\begin{frame}
\titlepage
\end{frame}


\begin{frame}{Актуальность темы}

\begin{itemize}
	\item повседневное использование людьми сжатых изображений
	\item 
\end{itemize}

\end{frame}


\begin{frame}{Методы сжатия}
\begin{itemize}
	\item Цветовое пространство $YC_bC_r$ и цветовая субдискретизация
	\item Квантование и дискретизация
	\item Вейвлет сжатие
	\item RLE
	\item Алгоритм сжатия LZW
	\item Алгоритм Хаффмана
	\item \dots
\end{itemize}
\end{frame}

\begin{frame}{Цветовое пространство $YC_bC_r$ и цветовая субдискретизация}

\begin{center}
	$Y = k_rR + k_gG + k_bB$

	$C_b = B - Y$ 

	$C_r = C - Y$ 

	$C_g = G - Y$ 
			
\end{center}
	
\begin{figure}[H]
	\begin{center}
		\includegraphics[scale=0.2]{../pics/YCbCr/YCbCr_separation_h.jpg}
		\caption{Цветное изображение и его компоненты Y, $C_B$ и $C_R$} 
		\label{pic:YCbCr:separation}
	\end{center}
\end{figure}	
		
\end{frame}

\begin{frame}{Блоки}

\begin{block}{Заголовок блока}
	Текст блока примера
\end{block}

\begin{alertblock}{}
	Текст блока предупреждения 
\end{alertblock}

\end{frame}

\begin{frame}{Рисунок}
\begin{figure}[H]
	\includegraphics[scale=0.4]{../pics/shakal.png}
	\label{fig:shakal}
\end{figure}
\center{Текст под рисунком, не подпись}
\end{frame}

\begin{frame}{Рисунок с позиционированием (красиво, но неудобно)}
\begin{picture}(340,250)
	\put(-40,18){
		\includegraphics[width=0.5\textwidth]{../pics/shakal.png}
	}
	% \put(-10,18){\line(1,0){362}}	% Ориентировочные линии
	% \put(-10,18){\line(0,1){235}}
	% \put(-10,253){\line(1,0){362}}
	% \put(352,18){\line(0,1){235}}
	\put(175,130){
		% \fbox{ % Оборачивает рамкой для более точного позиционирования (есть погрешность)
			\begin{minipage}[t]{0.5\textwidth}
				\begin{itemize}
				\item Какое-то перечисление
				\item Внизу справа
				\item При использовании \emph{picture}
				\item Им придется размещать все элементы слайда
				\item Иначе результат будет трудно предсказать
				\end{itemize}
			\end{minipage}
		% }
	}	
	\put(175,250){
		% \fbox{ % Оборачивает рамкой для более точного позиционирования (есть погрешность)
			\begin{minipage}[t]{0.5\textwidth}
				\begin{figure}[H]
				\centering
				\includegraphics[width=\textwidth]{../pics/shakal.png}
				\label{fig:shakal}
				\end{figure}
				\center{Начало координат снизу слева}
			\end{minipage}
		% }
	}
\end{picture}
\end{frame}

\begin{frame}{Формулы, сложна}
Алфавитный подход кодирования растровых изображений
\begin{displaymath}
	I = M \cdot \log_{2}{N}
\end{displaymath}
$M$ -- количество пикселей\\
$N$ -- количество цветов в палитре
\end{frame}


\end{document}
